% Retira espaço extra obsoleto entre as frases.
\frenchspacing
\imprimircapa
% ---

% ---
% Folha de rosto
% (o * indica que haverá a ficha bibliográfica)
% ---
%\imprimirfolhaderosto*


% \begin{folhadeaprovacao}%

% \noindent Dissertação de autoria de Fulano de Tal, sob o título \textbf{``\imprimirtitulo''}, apresentada à Escola de Artes, Ciências e Humanidades da Universidade de São Paulo, para obtenção do título de Mestre em Ciências pelo Programa de Pós-graduação em Sistemas de Informação, na área de concentração Metodologia e Técnicas da Computação, aprovada em \_\_\_\_\_\_\_ de \_\_\_\_\_\_\_\_\_\_\_\_\_\_\_\_\_\_\_\_\_\_ de \_\_\_\_\_\_\_\_\_\_ pela comissão julgadora constituída pelos doutores:

% \vspace*{3cm}

% \begin{center}
% %-------------------------------------------------------------------------
% % Comentário adicional do PPgSI - Informações sobre ``assinaturas'':
% %
% % Para Qualificação e para versão original de Dissertação: deixar em 
% % branco (ou seja, assim como está abaixo), pois os membros da banca podem
% % mudar, mesmo que eles já estejam previstos.
% % 
% % Para versão corrigida de Dissertação: usar os dados dos examinadores que 
% % efetivamente participaram da defesa. 
% % 
% % Para versão corrigida de Dissertação: em caso de ``professora'', trocar 
% % por ``Profa. Dra.'' 
% % 
% % Para versão corrigida de Dissertação: ao colocar os nomes dos 
% % examinadores, remover o sublinhado
% % 
% % Para versão corrigida de Dissertação: ao colocar os nomes dos 
% % examinadores, usar seus nomes completos, exatamente conforme constam em 
% % seus Currículos Lattes
% % 
% % Para versão corrigida de Dissertação: ao colocar os nomes das 
% % instituições, remover o sublinhado e remover a palavra ``Instituição:''
% %
% % Não abreviar os nomes das instituições.
% %
% %-------------------------------------------------------------------------
% \_\_\_\_\_\_\_\_\_\_\_\_\_\_\_\_\_\_\_\_\_\_\_\_\_\_\_\_\_\_\_\_\_\_\_\_\_\_\_\_\_\_\_\_\_\_\_\_\_\_\_\_\_\_\_\_
% \vspace*{0.2cm} 
% \\ \textbf{Prof. Dr. \_\_\_\_\_\_\_\_\_\_\_\_\_\_\_\_\_\_\_\_\_\_\_\_\_\_\_\_\_\_\_\_\_\_\_\_\_\_\_\_\_\_\_\_\_\_\_\_\_\_\_\_\_\_\_\_\_\_\_\_\_\_} 
% \\ \vspace*{0.2cm} 
% Instituição: \_\_\_\_\_\_\_\_\_\_\_\_\_\_\_\_\_\_\_\_\_\_\_\_\_\_\_\_\_\_\_\_\_\_\_\_\_\_\_\_\_\_\_\_\_\_\_\_\_\_\_\_\_\_\_\_\_\_ 
% \\ \vspace*{0.2cm}
% Presidente 

% \vspace*{2cm}

% \_\_\_\_\_\_\_\_\_\_\_\_\_\_\_\_\_\_\_\_\_\_\_\_\_\_\_\_\_\_\_\_\_\_\_\_\_\_\_\_\_\_\_\_\_\_\_\_\_\_\_\_\_\_\_\_
% \vspace*{0.2cm} 
% \\ \textbf{Prof. Dr. \_\_\_\_\_\_\_\_\_\_\_\_\_\_\_\_\_\_\_\_\_\_\_\_\_\_\_\_\_\_\_\_\_\_\_\_\_\_\_\_\_\_\_\_\_\_\_\_\_\_\_\_\_\_\_\_\_\_\_\_\_\_} 
% \\ \vspace*{0.2cm} 
% Instituição: \_\_\_\_\_\_\_\_\_\_\_\_\_\_\_\_\_\_\_\_\_\_\_\_\_\_\_\_\_\_\_\_\_\_\_\_\_\_\_\_\_\_\_\_\_\_\_\_\_\_\_\_\_\_\_\_\_\_

% \vspace*{2cm}

% \_\_\_\_\_\_\_\_\_\_\_\_\_\_\_\_\_\_\_\_\_\_\_\_\_\_\_\_\_\_\_\_\_\_\_\_\_\_\_\_\_\_\_\_\_\_\_\_\_\_\_\_\_\_\_\_
% \vspace*{0.2cm} 
% \\ \textbf{Prof. Dr. \_\_\_\_\_\_\_\_\_\_\_\_\_\_\_\_\_\_\_\_\_\_\_\_\_\_\_\_\_\_\_\_\_\_\_\_\_\_\_\_\_\_\_\_\_\_\_\_\_\_\_\_\_\_\_\_\_\_\_\_\_\_} 
% \\ \vspace*{0.2cm} 
% Instituição: \_\_\_\_\_\_\_\_\_\_\_\_\_\_\_\_\_\_\_\_\_\_\_\_\_\_\_\_\_\_\_\_\_\_\_\_\_\_\_\_\_\_\_\_\_\_\_\_\_\_\_\_\_\_\_\_\_\_

% \vspace*{2cm}

% \_\_\_\_\_\_\_\_\_\_\_\_\_\_\_\_\_\_\_\_\_\_\_\_\_\_\_\_\_\_\_\_\_\_\_\_\_\_\_\_\_\_\_\_\_\_\_\_\_\_\_\_\_\_\_\_
% \vspace*{0.2cm} 
% \\ \textbf{Prof. Dr. \_\_\_\_\_\_\_\_\_\_\_\_\_\_\_\_\_\_\_\_\_\_\_\_\_\_\_\_\_\_\_\_\_\_\_\_\_\_\_\_\_\_\_\_\_\_\_\_\_\_\_\_\_\_\_\_\_\_\_\_\_\_} 
% \\ \vspace*{0.2cm} 
% Instituição: \_\_\_\_\_\_\_\_\_\_\_\_\_\_\_\_\_\_\_\_\_\_\_\_\_\_\_\_\_\_\_\_\_\_\_\_\_\_\_\_\_\_\_\_\_\_\_\_\_\_\_\_\_\_\_\_\_\_

% \end{center}
  
% \end{folhadeaprovacao}
% % ---

% % ---
% % Dedicatória
% % ---
% %-------------------------------------------------------------------------
% % Comentário adicional do PPgSI - Informações sobre ``Dedicatória'': 
% %
% % Opcional para Dissertação.
% % Não sugerido para Qualificação.
% % 
% %-------------------------------------------------------------------------
% \begin{dedicatoria}
%    \vspace*{\fill}
%    \centering
%    \noindent
%    \textit{Escreva aqui sua dedicatória, se desejar, ou remova esta página...} 
% 	 \vspace*{\fill}
% \end{dedicatoria}
% ---

% ---
% Agradecimentos
% ---
%-------------------------------------------------------------------------
% Comentário adicional do PPgSI - Informações sobre ``Agradecimentos'': 
%
% Opcional para Dissertação.
% Não sugerido para Qualificação.
% 
% Lembrar de agradecer agências de fomento e outras instituições similares.
%
%-------------------------------------------------------------------------
% \begin{agradecimentos}
% Texto de exemplo, texto de exemplo, texto de exemplo, texto de exemplo, texto de exemplo, texto de exemplo, texto de exemplo, texto de exemplo, texto de exemplo, texto de exemplo, texto de exemplo, texto de exemplo, texto de exemplo, texto de exemplo, texto de exemplo, texto de exemplo, texto de exemplo, texto de exemplo, texto de exemplo, texto de exemplo, texto de exemplo, texto de exemplo.

% Texto de exemplo, texto de exemplo, texto de exemplo, texto de exemplo, texto de exemplo, texto de exemplo, texto de exemplo, texto de exemplo, texto de exemplo, texto de exemplo, texto de exemplo, texto de exemplo, texto de exemplo, texto de exemplo, texto de exemplo, texto de exemplo, texto de exemplo, texto de exemplo, texto de exemplo, texto de exemplo, texto de exemplo, texto de exemplo.

% Texto de exemplo, texto de exemplo, texto de exemplo, texto de exemplo, texto de exemplo, texto de exemplo, texto de exemplo, texto de exemplo, texto de exemplo, texto de exemplo, texto de exemplo, texto de exemplo, texto de exemplo, texto de exemplo, texto de exemplo, texto de exemplo, texto de exemplo, texto de exemplo, texto de exemplo, texto de exemplo, texto de exemplo, texto de exemplo.

% Texto de exemplo, texto de exemplo, texto de exemplo, texto de exemplo, texto de exemplo, texto de exemplo, texto de exemplo, texto de exemplo, texto de exemplo, texto de exemplo, texto de exemplo, texto de exemplo, texto de exemplo, texto de exemplo, texto de exemplo, texto de exemplo, texto de exemplo, texto de exemplo, texto de exemplo, texto de exemplo, texto de exemplo, texto de exemplo.

% Texto de exemplo, texto de exemplo, texto de exemplo, texto de exemplo, texto de exemplo, texto de exemplo, texto de exemplo, texto de exemplo, texto de exemplo, texto de exemplo, texto de exemplo, texto de exemplo, texto de exemplo, texto de exemplo, texto de exemplo, texto de exemplo, texto de exemplo, texto de exemplo, texto de exemplo, texto de exemplo, texto de exemplo, texto de exemplo.
% \end{agradecimentos}
% ---

% ---
% Epígrafe
% ---
%-------------------------------------------------------------------------
% Comentário adicional do PPgSI - Informações sobre ``Epígrafe'': 
%
% Opcional para Dissertação.
% Não sugerido para Qualificação.
% 
%-------------------------------------------------------------------------
% \begin{epigrafe}
%     \vspace*{\fill}
% 	\begin{flushright}
% 		\textit{``Escreva aqui uma epígrafe, se desejar, ou remova esta página...''\\
% 		(Autor da epígrafe)}
% 	\end{flushright}
% \end{epigrafe}
% % ---

% ---
% RESUMOS
% ---

% % resumo em português
% \setlength{\absparsep}{18pt} % ajusta o espaçamento dos parágrafos do resumo
% \begin{resumo}

% %-------------------------------------------------------------------------
% % Comentário adicional do PPgSI - Informações sobre ``referência'':
% % 
% % Troque os seguintes campos pelos dados de sua Dissertação (mantendo a 
% % formatação e pontuação):
% %   - SOBRENOME
% %   - Nome1
% %   - Nome2
% %   - Nome3
% %   - Título do trabalho: subtítulo do trabalho
% %   - AnoDeDefesa
% %
% % Mantenha todas as demais informações exatamente como estão.
% % 
% % [Não usar essas informações de ``referência'' para Qualificação]
% %
% %-------------------------------------------------------------------------
% \begin{flushleft}
% SOBRENOME, Nome1 Nome2 Nome3. \textbf{Título do trabalho}: subtítulo do trabalho. \imprimirdata. \pageref{LastPage} f. Dissertação (Mestrado em Ciências) – Escola de Artes, Ciências e Humanidades, Universidade de São Paulo, São Paulo, AnoDeDefesa.
% \end{flushleft}

% Escreva aqui o texto do seu resumo... (redigido em parágrafo único, no máximo em uma página, contendo no ``máximo 500 palavras'', e apresentando um resumo de todos o seu trabalho, incluindo objetivos, metodologia, resultados e conclusões; não inclua apenas a contextualização até chegar nos objetivos, é importante fazer um resumo de todos os capítulos do texto, até chegar à conclusão). Texto de exemplo, texto de exemplo, texto de exemplo, texto de exemplo, texto de exemplo, texto de exemplo, texto de exemplo, texto de exemplo, texto de exemplo, texto de exemplo, texto de exemplo, texto de exemplo, texto de exemplo, texto de exemplo, texto de exemplo, texto de exemplo, texto de exemplo, texto de exemplo, texto de exemplo, texto de exemplo, texto de exemplo, texto de exemplo.

% Palavras-chaves: Palavra1. Palavra2. Palavra3. etc.
% \end{resumo}

% % resumo em inglês
% %-------------------------------------------------------------------------
% % Comentário adicional do PPgSI - Informações sobre ``resumo em inglês''
% % 
% % Caso a Qualificação ou a Dissertação inteira seja elaborada no idioma inglês, 
% % então o ``Abstract'' vem antes do ``Resumo''.
% % 
% %-------------------------------------------------------------------------
% \begin{resumo}[Abstract]
% \begin{otherlanguage*}{english}

% %-------------------------------------------------------------------------
% % Comentário adicional do PPgSI - Informações sobre ``referência em inglês''
% % 
% % Troque os seguintes campos pelos dados de sua Dissertação (mantendo a 
% % formatação e pontuação):
% %     - SURNAME
% %     - FirstName1
% %     - MiddleName1
% %     - MiddleName2
% %     - Work title: work subtitle
% %     - DefenseYear (Ano de Defesa)
% %
% % Mantenha todas as demais informações exatamente como estão.
% %
% % [Não usar essas informações de ``referência'' para Qualificação]
% %
% %-------------------------------------------------------------------------
% \begin{flushleft}
% SURNAME, FirstName MiddleName1 MiddleName2. \textbf{Work title}: work subtitle. \imprimirdata. \pageref{LastPage} p. Dissertation (Master of Science) – School of Arts, Sciences and Humanities, University of São Paulo, São Paulo, DefenseYear. 
% \end{flushleft}

% Write here the English version of your ``Resumo''. Example text, example text, example text, example text, example text, example text, example text, example text, example text, example text, example text, example text, example text, example text, example text, example text, example text, example text, example text, example text, example text, example text, example text, example text, example text, example text, example text, example text, example text, example text, example text, example text, example text, example text, example text, example text, example text, example text, example text, example text, example text, example text, example text, example text, example text, example text, example text.

% Keywords: Keyword1. Keyword2. Keyword3. etc.
% \end{otherlanguage*}
% \end{resumo}

% % ---
% % ---
% % inserir lista de figuras
% % ---
% \pdfbookmark[0]{\listfigurename}{lof}
% \listoffigures*
% \cleardoublepage
% % ---

% % ---
% % inserir lista de algoritmos
% % ---
% \pdfbookmark[0]{\listalgorithmname}{loa}
% \listofalgorithms
% \cleardoublepage

% % ---
% % inserir lista de quadros
% % ---
% \pdfbookmark[0]{\listofquadrosname}{loq}
% \listofquadros*
% \cleardoublepage


% % ---
% % inserir lista de tabelas
% % ---
% \pdfbookmark[0]{\listtablename}{lot}
% \listoftables*
% \cleardoublepage
% % ---

% % ---
% % inserir lista de abreviaturas e siglas
% % ---
% %-------------------------------------------------------------------------
% % Comentário adicional do PPgSI - Informações sobre ``Lista de abreviaturas 
% % e siglas'': 
% %
% % Opcional.
% % Uma vez que se deseja usar, é necessário manter padrão e consistência no
% % trabalho inteiro.
% % Se usar: inserir em ordem alfabética.
% %
% %-------------------------------------------------------------------------
% \begin{siglas}
%   \item[Sigla/abreviatura 1] Definição da sigla ou da abreviatura por extenso
%   \item[Sigla/abreviatura 2] Definição da sigla ou da abreviatura por extenso
%   \item[Sigla/abreviatura 3] Definição da sigla ou da abreviatura por extenso
%   \item[Sigla/abreviatura 4] Definição da sigla ou da abreviatura por extenso
%   \item[Sigla/abreviatura 5] Definição da sigla ou da abreviatura por extenso
%   \item[Sigla/abreviatura 6] Definição da sigla ou da abreviatura por extenso
%   \item[Sigla/abreviatura 7] Definição da sigla ou da abreviatura por extenso
%   \item[Sigla/abreviatura 8] Definição da sigla ou da abreviatura por extenso
%   \item[Sigla/abreviatura 9] Definição da sigla ou da abreviatura por extenso
%   \item[Sigla/abreviatura 10] Definição da sigla ou da abreviatura por extenso
% \end{siglas}
% % ---

% % ---
% % inserir lista de símbolos
% % ---
% %-------------------------------------------------------------------------
% % Comentário adicional do PPgSI - Informações sobre ``Lista de símbolos'': 
% %
% % Opcional.
% % Uma vez que se deseja usar, é necessário manter padrão e consistência no
% % trabalho inteiro.
% % Se usar: inserir na ordem em que aparece no texto.
% % 
% %-------------------------------------------------------------------------
% \begin{simbolos}
%   \item[$ \Gamma $] Letra grega Gama
%   \item[$ \Lambda $] Lambda
%   \item[$ \zeta $] Letra grega minúscula zeta
%   \item[$ \in $] Pertence
% \end{simbolos}
% % ---

% % ---
% % inserir o sumario
% % ---
\pdfbookmark[0]{\contentsname}{toc}
\tableofcontents*
\cleardoublepage
% % ---

% % ---
% % compila o indice
% % ---
% % ---



% %-------------------------------------------------------------------------
% % Comentário adicional do PPgSI - Informações sobre ``títulos de seções''
% % 
% % Para todos os títulos (seções, subseções, tabelas, ilustrações, etc.):
% %
% % Em maiúscula apenas a primeira letra da sentença (do título), exceto 
% % nomes próprios, geográficos, institucionais ou Programas ou Projetos ou
% % siglas, os quais podem ter letras em maiúscula também.
% %
% %-------------------------------------------------------------------------