\chapter{Chamadas ao Sistema Operacional}

O conjunto de instruções MIPS também prevê um conjunto de chamadas ao sistema operacional que podem ser invocadas através do comando $syscall$. Ao invocar o comando $syscall$, deve-se preencher alguns registradores de modo a indicar qual ação deve ser executada e quais são os argumentos para a ação. O registrador $\$v0$ indicará qual ação deve ser executada e possíveis argumentos são armazenados nos registradores $\$a0-\$a3$. Quando o sistema operacional executa a ação invocada pelo $syscall$, dependendo da ação, se gerar retorno, este pode ser armazenado nos registradores $\$v0$ e $\$v1$, dedicados a isso.

As chamadas suportadas pelo simulador MARS, bem como quais argumentos são esperados e quais retornos são devolvidos estão listadas abaixo:\\

\begin{table}[H]
  \centering
  \caption{Chamadas ao Sistema Operacional suportadas pelo MARS}
	\label{tab:chamadas}
  \begin{tabu} to 1.0\textwidth {|X[c,m]|X[c,m]|X[l,m]|X[l,m]|}
   \hline
\multicolumn{1}{|c}{\textbf{Função}} & 
\multicolumn{1}{|c}{\textbf{Código em \$v0}} & 
\multicolumn{1}{|c}{\textbf{Argumentos}} & 
\multicolumn{1}{|c|}{\textbf{Resultado}}\\\hline
Imprimir inteiro (integer) \vspace*{0.2cm} & 1 & \$a0 = inteiro a ser impresso &  \\ \hline
Imprimir ponto flutuante (float) & 2 & \$f12 = ponto flutuante a ser impresso &  \\ \hline
\vspace*{0.2cm} Imprimir ponto flutuante de precisão dupla (double) \vspace*{0.2cm} & 3 & \$f12 = ponto flutuante a ser impresso &  \\ \hline
\vspace*{0.2cm} Imprimir cadeia de caracteres (string) \vspace*{0.2cm} & 4 & \$a0 = endereço da cadeia de caracteres terminada em null a ser impressa &  \\ \hline
\vspace*{0.2cm} Ler inteiro da entrada padrão \vspace*{0.2cm} & 5 &  & \$v0 contém o inteiro lido \\ \hline
\vspace*{0.2cm} Ler ponto flutuante da entrada padrão \vspace*{0.2cm} & 6 &  & \$f0 contém o ponto flutuante lido \\ \hline
\vspace*{0.2cm} Ler ponto flutuante de dupla precisão da entrada padrão \vspace*{0.2cm} & 7 &  & \$f0 contém o ponto flutuante lido \\ \hline
\end{tabu}
\end{table}

\begin{table}[H]
  \centering
  \begin{tabu} to 1.0\textwidth {|X[c,m]|X[c,m]|X[l,m]|X[l,m]|}
   \hline
\multicolumn{1}{|c}{\textbf{Função}} & 
\multicolumn{1}{|c}{\textbf{Código em \$v0}} & 
\multicolumn{1}{|c}{\textbf{Argumentos}} & 
\multicolumn{1}{|c|}{\textbf{Resultado}}\\\hline
Ler cadeia de caracteres da entrada padrão & 8 & \vspace*{0.2cm} \$a0 = endereço do buffer de entrada \newline \$a1 = quantidade máxima de caracteres a serem lidos \vspace*{0.2cm} & Ver notas abaixo \\ \hline
sbrk (alocar memória da pilha) & 9 & \vspace*{0.2cm} \$a0 = quantidade de bytes a serem alocados \vspace*{0.2cm} & \$v0 contém o endereço da memória alocada   \\ \hline
\vspace*{0.2cm} Sair (terminar execução) \vspace*{0.2cm}& 10 &  &  \\ \hline
Imprimir caractere (char) & 11 & \vspace*{0.2cm} \$a0 = caractere a ser impresso \vspace*{0.2cm} & Veja notas abaixo \\ \hline
\vspace*{0.2cm} Ler caractere da entrada padrão \vspace*{0.2cm} & 12 &  & \$v0 contém o caractere lido \\ \hline
Abrir arquivo & 13 & \vspace*{0.2cm} \$a0 = endereço da cadeia de caracteres terminadas em null contendo o caminho do arquivo \newline \$a1 = flags \newline \$a2 = modo de abertura \vspace*{0.2cm} & \$v0 contém o descritor do arquivo (ponteiro –negativo se falhar). \newline Veja notas abaixo \\ \hline
Ler do arquivo & 14 & \vspace*{0.2cm} \$a0 = descritor do arquivo \newline \$a1 = endereço do buffer de entrada \newline \$a2 = quantidade máxima de caracteres a serem lidos \vspace*{0.2cm} & \$v0 contém a quantidade de caracteres lidos (0 se final de arquivo; negativo se falhar) \\ \hline
Escrever em arquivo & 15 & \vspace*{0.2cm} \$a0 = descritor do arquivo \newline \$a1 = endereço do buffer de entrada\$a2 = quantidade máxima de caracteres a serem lidos \vspace*{0.2cm} & \$v0 contém a quantidade de caracteres lidos (negativo se falhar) \\ \hline
Fechar arquivo & 16 & \vspace*{0.2cm} \$a0 = descritor do arquivo \vspace*{0.2cm} &  \\ \hline
\vspace*{0.2cm} Sair (terminar com valor) \vspace*{0.2cm} & 17 & \$a0 = resultado do término & Veja notas abaixo \\ \hline
   \end{tabu}
\end{table}

\newpage
\textbf{Notas:}\\

\textbf{Função 8} -- Segue a semântica da função $fgets$ do UNIX. Para um determinado tamanho $n$, a cadeia de caracteres não pode ser superior a $n - 1$. Se menos do que isso, adiciona uma quebra de linha ao fim. Em qualquer caso, preenche com o $byte null$. Se $n = 1$, a entrada é ignorada e apenas o $byte null$ é armazenado no $buffer$. Se $n < 1$, a entrada é ignorada e nada é escrito no $buffer$.

\textbf{Função 11} -- Imprime o caractere ASCII correspondente ao $byte$ de menor ordem.

\textbf{Função 13} – O MARS implementa três valores de flags: 0 para `apenas leitura', 1 para `apenas escrita com criação' e 9 para `apenas escrita com criação e junção'. Essa função ignora o valor do modo. O descritor de arquivo retornado será negativo se a operação falhar. A implementação de entrada e saída de arquivos por baixo usa os métodos $java.io.FileInputStream.read()$ para ler e $java.io.FileOutputStream.write()$ para escrever. O MARS mantém descritores de arquivos internamente e os aloca começando com 3. Descritores 0, 1 e 2 estão sempre abertos para: ler da entrada padrão, escrever para a saída padrão, e escrever para a saída de erros padrão.

\textbf{Função 17} – Se o programa MIPS é executado sob controle da interface gráfica MARS (GUI), o código de saída em $\$a0$ é ignorado.
